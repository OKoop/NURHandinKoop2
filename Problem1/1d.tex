\subsection*{1d}

We read in the sample from 1c. Now we perform a Kuiper test. We use astropy to check the Kuiper test against.\\

Our script for this question is given by:
\lstinputlisting{./Problem1/Handin21d.py}

Our script produces the following output in .txt:
{\obeylines\obeyspaces
\texttt{
\input{./Problem1/outputs1d.txt}
}}

It produces the following figures:
\begin{figure}[H]
  \centering
  \includegraphics[width=0.75\linewidth]{./Problem1/ds1d.png}
  \caption{\textit{These are the astropy- and my own values of the Kuiper-statistic (d). As you can see they are identical.}}
\end{figure}

\begin{figure}[H]
  \centering
  \includegraphics[width=0.75\linewidth]{./Problem1/ps1d.png}
  \caption{\textit{These are the p-values of my own and of astropy's algorithm. The red line is an arbitrary threshold-value. It signifies the minimum probability at which the distributions are consistent.}}
\end{figure}

The p-value oscillates a bit, but is above the given threshold all the time now. 

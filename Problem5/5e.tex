\subsection*{5e}

By generalizing the 1d FFT from 5d we get a 2- and 3D-FFT for use in this exercise.\\
I generalized the sine from 5d for 2D and implemented a multvariate Gaussian (of which we expect just a multivariate Gaussian as FFT again).

The code used for running the problem is:
\lstinputlisting{./Problem5/Handin25e.py}

It produced the following output in .txt:
{\obeylines\obeyspaces
\texttt{
\input{./Problem5/outputs5e.txt}
}}

It produced the following figures:
\begin{figure}[H]
  \centering
  \includegraphics[width=0.75\linewidth]{./Problem5/e2down.png}
  \caption{\textit{FFT of $f(x,y)=\sin(2\pi (x+y)/5)$ for my implementation.}}
\end{figure}

\begin{figure}[H]
  \centering
  \includegraphics[width=0.75\linewidth]{./Problem5/e2dscipy.png}
  \caption{\textit{FFT of $f(x,y)=\sin(2\pi (x+y)/5)$ for scipy.fftpack.}}
\end{figure}

The theoretical FFT is equal to only the two yellow points shown in the figures above.

\begin{figure}[H]
  \centering
  \includegraphics[width=0.75\linewidth]{./Problem5/e3dxzscipy.png}
  \caption{\textit{FFT of a multivariate Gaussian in 3D in the $x-z$-plane for scipy.fftpack.}}
\end{figure}

\begin{figure}[H]
  \centering
  \includegraphics[width=0.75\linewidth]{./Problem5/e3dxyscipy.png}
  \caption{\textit{FFT of a multivariate Gaussian in 3D in the $x-y$-plane for scipy.fftpack.}}
\end{figure}

\begin{figure}[H]
  \centering
  \includegraphics[width=0.75\linewidth]{./Problem5/e3dyzscipy.png}
  \caption{\textit{FFT of a multivariate Gaussian in 3D in the $y-z$-plane for scipy.fftpack.}}
\end{figure}

\begin{figure}[H]
  \centering
  \includegraphics[width=0.75\linewidth]{./Problem5/e3dxzown.png}
  \caption{\textit{FFT of a multivariate Gaussian in 3D in the $x-z$-plane for my own implementation.}}
\end{figure}

\begin{figure}[H]
  \centering
  \includegraphics[width=0.75\linewidth]{./Problem5/e3dxyown.png}
  \caption{\textit{FFT of a multivariate Gaussian in 3D in the $x-y$-plane for my own implementation.}}
\end{figure}

\begin{figure}[H]
  \centering
  \includegraphics[width=0.75\linewidth]{./Problem5/e3dyzown.png}
  \caption{\textit{FFT of a multivariate Gaussian in 3D in the $y-z$-plane for my own implementation.}}
\end{figure}

As said above this is as we expected.


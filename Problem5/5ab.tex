\subsection*{5ab}

If the shape is a delta function with a given mass, and we take the mass and size to be of unit value (physical unit not important) we get the results found below. I combined 5a and b because they depend on each other.\\

To find the grid point closest to a particle we add $.5$ and make an int out of it, because that rounds it down always. We thus assign the mass of the particle to the grid point closest to it by adding 1 to the cell signified by the hereby found coordinates.\\

The code used for running the problem is:
\lstinputlisting{./Problem5/Handin25ab.py}

It produced the following output in .txt:
{\obeylines\obeyspaces
\texttt{
\input{./Problem5/outputs5ab.txt}
}}

It produced the following figures:
\begin{figure}[H]
  \centering
  \includegraphics[width=0.75\linewidth]{./Problem5/a4.png}
  \caption{\textit{The slice for $z=4$ of the NGP-method mesh.}}
\end{figure}

\begin{figure}[H]
  \centering
  \includegraphics[width=0.75\linewidth]{./Problem5/a9.png}
  \caption{\textit{The slice for $z=4$ of the NGP-method mesh.}}
\end{figure}

\begin{figure}[H]
  \centering
  \includegraphics[width=0.75\linewidth]{./Problem5/a11.png}
  \caption{\textit{The slice for $z=4$ of the NGP-method mesh.}}
\end{figure}

\begin{figure}[H]
  \centering
  \includegraphics[width=0.75\linewidth]{./Problem5/a14.png}
  \caption{\textit{The slice for $z=4$ of the NGP-method mesh.}}
\end{figure}

\begin{figure}[H]
  \centering
  \includegraphics[width=0.75\linewidth]{./Problem5/b4.png}
  \caption{\textit{The value of the bin for the 4th cell for one particle at all possible positions.}}
\end{figure}

\begin{figure}[H]
  \centering
  \includegraphics[width=0.75\linewidth]{./Problem5/b0.png}
  \caption{\textit{The value of the bin for the 0th cell for one particle at all possible positions.}}
\end{figure}

The last two figures look as I'd have expected given the method we use (and the periodic boundary conditions).
\subsection*{5c}

We assign mass now by checking which parts of the cloud are closest to which grid points. This can be done by getting the floor values of the coordinates of a given particle (this will be the bottom left corner of the cell containing the particle), and finding the distance along each direction from the particle to that floor-point.\\
These are the height, depth and length of the 'subbox' of the box around the particle that is closest to the grid point opposite to the bottom left corner.\\
(This is way more easily explained graphically....).\\
In the same way we can assign mass to the other grid points along this cell.

The code used for running the problem is:
\lstinputlisting{./Problem5/Handin25c.py}

It produced the following output in .txt:
{\obeylines\obeyspaces
\texttt{
\input{./Problem5/outputs5c.txt}
}}

It produced the following figures:
\begin{figure}[H]
  \centering
  \includegraphics[width=0.75\linewidth]{./Problem5/c4.png}
  \caption{\textit{The slice for $z=4$ of the CIC-method mesh.}}
\end{figure}

\begin{figure}[H]
  \centering
  \includegraphics[width=0.75\linewidth]{./Problem5/c9.png}
  \caption{\textit{The slice for $z=4$ of the CIC-method mesh.}}
\end{figure}

\begin{figure}[H]
  \centering
  \includegraphics[width=0.75\linewidth]{./Problem5/c11.png}
  \caption{\textit{The slice for $z=4$ of the CIC-method mesh.}}
\end{figure}

\begin{figure}[H]
  \centering
  \includegraphics[width=0.75\linewidth]{./Problem5/c14.png}
  \caption{\textit{The slice for $z=4$ of the CIC-method mesh.}}
\end{figure}

\begin{figure}[H]
  \centering
  \includegraphics[width=0.75\linewidth]{./Problem5/c4c.png}
  \caption{\textit{The value of the bin for the 4th cell for one particle at all possible positions.}}
\end{figure}

\begin{figure}[H]
  \centering
  \includegraphics[width=0.75\linewidth]{./Problem5/c0c.png}
  \caption{\textit{The value of the bin for the 0th cell for one particle at all possible positions.}}
\end{figure}

These first figures look like a smoother counterpart to the figures from 5a, which was to be expected.\\
The last two figures again look as I'd have expected given the method we use (and the periodic boundary conditions).

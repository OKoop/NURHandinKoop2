\documentclass[10pt]{article}
\usepackage{amssymb}
\usepackage{amsmath}
\usepackage{float}
\usepackage[utf8]{inputenc}
\setlength{\textheight}{25.7cm}
\setlength{\textwidth}{16cm}
\setlength{\unitlength}{1mm}
\setlength{\topskip}{2.5truecm}
\topmargin 260mm \advance \topmargin -\textheight 
\divide \topmargin by 2 \advance \topmargin -1in 
\headheight 0pt \headsep 0pt \leftmargin 210mm \advance
\leftmargin -\textwidth 
\divide \leftmargin by 2 \advance \leftmargin -1in 
\oddsidemargin \leftmargin \evensidemargin \leftmargin
\parindent=0pt
\frenchspacing
\usepackage{microtype}
\usepackage{tikz}
\usepackage[english,dutch]{babel}
\usepackage{enumerate}
\usepackage{graphicx}
\usepackage[procnames]{listings}
\newcommand{\iso}{\ \raisebox{0.1ex}{\ensuremath{\stackrel{\sim}{\longrightarrow}}\ \,}}
\newcommand{\Z}{\mathbb{Z}}
\newcommand{\e}{\epsilon}
\renewcommand{\O}{\Omega}
\renewcommand{\o}{\omega}
\newcommand{\p}[2]{\dfrac{\partial #1}{\partial #2}}
\renewcommand{\a}{\alpha}
\renewcommand{\r}{\rho}
\renewcommand{\b}{\beta}
\newcommand{\D}{\Delta}
\newcommand{\s}{\sigma}
\newcommand{\g}{\gamma}
\newcommand{\G}{\Gamma}
\renewcommand{\d}{\delta}
\renewcommand{\l}{\lambda}
\renewcommand{\t}{\theta}
\renewcommand{\L}{\Lambda}
\newcommand{\ov}[1]{\overline{#1}}
\usepackage{amsmath}
\usepackage{fullpage}
\usepackage{hyperref}
\usepackage{graphicx}
\usepackage{listings}
\usepackage{color}

\definecolor{mygreen}{rgb}{0,0.6,0}
\definecolor{mygray}{rgb}{0.5,0.5,0.5}
\definecolor{mymauve}{rgb}{0.58,0,0.82}

\lstset{ 
  backgroundcolor=\color{white},   % choose the background color; you must add \usepackage{color} or \usepackage{xcolor}; should come as last argument
  basicstyle=\footnotesize,        % the size of the fonts that are used for the code
  breakatwhitespace=false,         % sets if automatic breaks should only happen at whitespace
  breaklines=true,                 % sets automatic line breaking
  captionpos=b,                    % sets the caption-position to bottom
  commentstyle=\color{mygreen},    % comment style
  deletekeywords={...},            % if you want to delete keywords from the given language
  escapeinside={\%*}{*)},          % if you want to add LaTeX within your code
  extendedchars=true,              % lets you use non-ASCII characters; for 8-bits encodings only, does not work with UTF-8
  firstnumber=1,                   % start line enumeration with line 1000
  frame=single,	                   % adds a frame around the code
  keepspaces=true,                 % keeps spaces in text, useful for keeping indentation of code (possibly needs columns=flexible)
  keywordstyle=\color{blue},       % keyword style
  language=Python,                 % the language of the code
  morekeywords={*,...},            % if you want to add more keywords to the set
  numbers=left,                    % where to put the line-numbers; possible values are (none, left, right)
  numbersep=5pt,                   % how far the line-numbers are from the code
  numberstyle=\tiny\color{mygray}, % the style that is used for the line-numbers
  rulecolor=\color{black},         % if not set, the frame-color may be changed on line-breaks within not-black text (e.g. comments (green here))
  showspaces=false,                % show spaces everywhere adding particular underscores; it overrides 'showstringspaces'
  showstringspaces=false,          % underline spaces within strings only
  showtabs=false,                  % show tabs within strings adding particular underscores
  stepnumber=1,                    % the step between two line-numbers. If it's 1, each line will be numbered
  stringstyle=\color{mymauve},     % string literal style
  tabsize=4,	                   % sets default tabsize to 2 spaces
  title=\lstname                   % show the filename of files included with \lstinputlisting; also try caption instead of title
}

\title{NUR Solutions Hand-In Set 2}
\author{Orlin Koop, s1676059}

\begin{document}

\maketitle

\begin{abstract}
In this document the solutions for the NUR second hand-in set are given. The Output of the python script per subquestion is shown, with commentary. We thus have sections and subsections for each exercise.
\end{abstract}

Note that when nothing shows up under the sentence: "It produces the following output in .txt", no output was generated.

\section{Normally distributed pseudo-random numbers}

Here we look at the answers to the subquestions from Problem 1 from the second Hand-in set. Note that I included the code I created for running 1b before it got deleted from the exercise. The general functions that I use for most of these exercises are coded up in a more general file given before the solutions:

Our script is given by:
\lstinputlisting{./Problem1/functions21.py}

It has to be stated that many of these algorithms are my own implementation of the C++-code included in 'Numerical Recipes'.\\
Now follow the answers to the subquestions:

\subsection*{1a}

We created a RNG which combines a MWC with a XOR-shift, in that combination. Because I made sure the RNG-state is a uint64 at every point in time I am sure this works fine.\\
Before this I had some problems with the normalization with sys.maxsize and with the randomness when I did not use uint64 as much as now.\\

Our script for this question is given by:
\lstinputlisting{./Problem1/Handin21a.py}

Our script produces the following output in .txt:
{\obeylines\obeyspaces
\texttt{
\input{./Problem1/outputs1a.txt}
}}
It thus states the seed used for questions 1 and 2. For question 4 we have set a new seed.

It produces the following figures:
\begin{figure}[H]
  \centering
  \includegraphics[width=0.75\linewidth]{./Problem1/hist1a.png}
  \caption{\textit{A histogram of 1 000 000 random numbers generated with the RNG implemented. It can be seen the RNG creates valid pseudo-random uniform samples.}}
\end{figure}

\begin{figure}[H]
  \centering
  \includegraphics[width=0.75\linewidth]{./Problem1/scatter1a1.png}
  \caption{\textit{Scatterplot of $x_{i+1}$ versus $x_i$ for the first 1000 generated random numbers. Here too we see that it creates valid pseudo-random numbers.}}
\end{figure}

\begin{figure}[H]
  \centering
  \includegraphics[width=0.75\linewidth]{./Problem1/scatter1a2.png}
  \caption{\textit{Scatterplot of the first 1000 random numbers versus their index (0 through 999). Again this shows the pseudo-randomness of the RNG is good enough for our purposes.}}
\end{figure}

\subsection*{1bthen}

I am not quite sure if everything works for this part, but I did code up some algorithms and almost all needed stuff when this was still a question. I used mostly algorithms I already had for Handin 1, and used the easiest, most robust ways of going about this exercise. The downhill simplex is implemented because I did not finish it for the last set and can now show that I got it working again. It is used to find the optimal parameters for the fitted line in logspace.\\

Our script for this question is given by:
\lstinputlisting{./Problem1/Handin21bthen.py}

Our script produces the following output in .txt:
{\obeylines\obeyspaces
\texttt{
\input{./Problem1/outputs1bt.txt}
}}
It gives the numerical coefficients of the line in logspace (see figure below). The analytical coefficients are: {\color{red}Do this if time left.}

It produces the following figures:
\begin{figure}[H]
  \centering
  \includegraphics[width=0.75\linewidth]{./Problem1/fits.png}
  \caption{\textit{Plot of the exact value of the integrated Gaussian for values of $x=2^i$ ($i=-10,-9,...,10$)(blue line) and the line through those points but when calculated with the Simpson integration algorithm.}}
\end{figure}

\subsection*{1bnow}

We read in the seed saved from 1a to continue producing random numbers.\\
The theoretical distribution is the Normed Gaussian given by:
$$NG(x,\mu,\s)=\dfrac{1}{\sqrt{2\pi\s^2}}e^{-\dfrac{(x-\mu)^2}{2\s^2}}.$$
In this case, $\mu=3$ and $\s=2.4$.

Our script for this question is given by:
\lstinputlisting{./Problem1/Handin21bnow.py}

Our script produces the following output in .txt:
{\obeylines\obeyspaces
\texttt{
\input{./Problem1/outputs1bn.txt}
}}

It produces the following figures:
\begin{figure}[H]
  \centering
  \includegraphics[width=0.75\linewidth]{./Problem1/1b.png}
  \caption{\textit{Histogram of RNG-generated BoxMuller normally distributed random numbers. Also a plot of a theoretical normalized gaussian of this distribution. The black lines specify the theoretical $\pm i\s$ intervals for $i=1,2,3,4,5$.}}
\end{figure}

The y-scale is logarithmic to be able to see the $-5\s$ and $5\s$ regions. I see that the outer regions do not have any values in the sample. This could be due to a few things: 1. A faulty RNG. 2. A mistake in binning. 3. The small size of the sample just not producing an 'outlier'.

\subsection*{1c}

We read in the seed saved from 1b to continue producing random numbers. We save the sample created here for use for the same end in question d and e. I implemented mergesort for sorting the data before performing the KS-test on it, because it is fast and easy to implement.\\
It should be noted that due to the way I implemented the BoxMuller method it can only deal with even sizes, so when you enter an odd number it will return an array with one less element. Therefore some elements from the sample created will not be used.

Our script for this question is given by:
\lstinputlisting{./Problem1/Handin21c.py}

Our script produces the following output in .txt:
{\obeylines\obeyspaces
\texttt{
\input{./Problem1/outputs1c.txt}
}}

It produces the following figures:
\begin{figure}[H]
  \centering
  \includegraphics[width=0.75\linewidth]{./Problem1/ds1c.png}
  \caption{\textit{These are the scipy- and my own values of the KS-statistic (d). As you can see they are identical.}}
\end{figure}

\begin{figure}[H]
  \centering
  \includegraphics[width=0.75\linewidth]{./Problem1/ps1c.png}
  \caption{\textit{These are the p-values of my own and of scipy's algorithm. The red line is an arbitrary threshold-value. The p-value signifies the probability that the distributions are consistent.}}
\end{figure}

The p-value oscillates a bit, but is above the given threshold most of the time (and especially for the biggest sample). 


\subsection*{1d}

We read in the sample from 1c. Now we perform a Kuiper test. We use astropy to check the Kuiper test against.\\

Our script for this question is given by:
\lstinputlisting{./Problem1/Handin21d.py}

Our script produces the following output in .txt:
{\obeylines\obeyspaces
\texttt{
\input{./Problem1/outputs1d.txt}
}}

It produces the following figures:
\begin{figure}[H]
  \centering
  \includegraphics[width=0.75\linewidth]{./Problem1/ds1d.png}
  \caption{\textit{These are the astropy- and my own values of the Kuiper-statistic (d). As you can see they are identical.}}
\end{figure}

\begin{figure}[H]
  \centering
  \includegraphics[width=0.75\linewidth]{./Problem1/ps1d.png}
  \caption{\textit{These are the p-values of my own and of astropy's algorithm. The red line is an arbitrary threshold-value. It signifies the minimum probability at which the distributions are consistent.}}
\end{figure}

The p-value oscillates a bit, but is above the given threshold all the time now. 


\subsection*{1e}

We read in the sample from 1c. Now we perform a KS-test for each of ten given samples.\\
To spare time I did not plot the scipy-answer because it was already proven my own implementation gave the same result.\\
I see now that running it on pczaal is faster than my laptop, so it is not needed.

Our script for this question is given by:
\lstinputlisting{./Problem1/Handin21e.py}

Our script produces the following output in .txt:
{\obeylines\obeyspaces
\texttt{
\input{./Problem1/outputs1e.txt}
}}

It produces the following figures:
\begin{figure}[H]
  \centering
  \hspace*{-1cm}
  \includegraphics[width=1.2\linewidth]{./Problem1/e.png}
  \caption{\textit{These are the scipy- and my own values of the KS-statistic (d). For 10 sets of random numbers, compared with a sample of normally distributed random numbers from my own RNG.}}
\end{figure}

For time efficiency I chose to plot it as subplots, but now it is not as clear anymore.\\ 
Still we can see that only the fourth set is consistent with my own RNG (BoxMuller method with $\mu=0$ and $\s=1$) according to the KS-test.





\section{Making an initial density field}

Here we look at the answers to Problem 2 from the second Hand-in set. The general functions that I use are coded up in a more general file given before the solutions. This is the basis for the functions used in Problem 4.

Our script is given by:
\lstinputlisting{./Problem2/functions22.py}

I do create my own array with ks-values which is a bit costly but I did not see another way.\\
I then iterate over only the top half of the 2D-array (including the Nyquist-band) because all those indexes are then mirrored into the bottom half when trying to get the array to be symmetric.\\
I generate all random numbers at once and then take them from this array to only call the RNG once during the entire program. Therefore you can clearly see the effect of $n$ in:
$\s^2=P(k)=k^n.$

The code used for running the problem is:
\lstinputlisting{./Problem2/Handin22.py}

It produced the following output in .txt:
{\obeylines\obeyspaces
\texttt{
\input{./Problem2/outputs2.txt}
}}

It produces the following figures:
\begin{figure}[H]
  \centering
  \includegraphics[width=0.75\linewidth]{./Problem2/n1.png}
  \caption{\textit{A GRF created in the Fourier space and then inversed Fourier transformed. The dispersion of the random complex numbers is $\s^2=k^{-1}$, where $k$ is the wavenumber. The axes are in Mpc.}}
\end{figure}

\begin{figure}[H]
  \centering
  \includegraphics[width=0.75\linewidth]{./Problem2/n2.png}
  \caption{\textit{The same as the last figure but with $\s^2=k^{-2}$.}}
\end{figure}

\begin{figure}[H]
  \centering
  \includegraphics[width=0.75\linewidth]{./Problem2/n3.png}
  \caption{\textit{The same as the last figure but with $\s^2=k^{-3}$.}}
\end{figure}

When the minimum size is equal to $1$Mpc (space of one pixel), the maximal size is that of the diagonal, thus $1448$Mpc, because two points at both sides of the diagonal create a wave of length 1448Mpc.\\
This means the minimal $k$ is equal to $0$, when both indices are $0$, because $k_j=\dfrac{2\pi}{N_g}\cdot i_j$, and $k=\sqrt{k_x^2+k_y^2}$. The maximal $k$ is thus equal to $k_x=2\pi/N_g\cdot N_g/2 = \pi$, thus $k=\sqrt{2}\pi$.

\section{Linear structure growth}

Here we look at the answers to Problem 3 from the second Hand-in set. The general functions that I use are coded up in a more general file given before the solutions. It mainly contains the ODE-er class, which can perform the Runge-Kutta 4th order algorithm for a second order ODE.\\
I implemented RK4 because it is the most simple and robust, and it works in this case.\\
Given $a$ we find:
\begin{align*}
a &= \left(\dfrac{3H_0t}{2}\right)^{2/3}\\
\dot{a} &= \dfrac{2}{3}\dfrac{3H_0}{2}\left(\dfrac{3 H_0 t}{2}\right)^{-1/3} = H_0\left(\dfrac{3 H_0 t}{2}\right)^{-1/3}\\
\dfrac{\dot{a}}{a} &= H_0\dfrac{2}{3 H_0 t} = \dfrac{2}{3t}\\
\dfrac{1}{a^3} &= \dfrac{4}{9H_0^2t^2}.
\end{align*}
Thus the 2nd order ODE becomes:
$$\dfrac{d^2D}{dt^2}+\dfrac{4}{3t}\dfrac{dD}{dt}=\dfrac{2}{3t^2}\Omega_0D.$$
We split the 2nd order ODE into a pair of coupled 1st order ODEs:
\begin{align*}
\dfrac{dD}{dt} &= z\\
\dfrac{dz}{dt} &= \dfrac{2}{3t^2}\Omega_0D - \dfrac{4}{3t}z
\end{align*}

The analytical solution of these, for $\Omega_0=\Omega_m=1$, is given as:
\begin{align*}
D(t) &= C_1t^{2/3}+C_2/t\\
C_1 &= \dfrac{3}{5}(D(1)+D'(1))\\
C_2 &= \dfrac{2}{5}\left(D(1) - \dfrac{3}{2}D'(1)\right)
\end{align*}

Our script is given by:
\lstinputlisting{./Problem3/functions23.py}


The code used for running the problem is:
\lstinputlisting{./Problem3/Handin23.py}

It produced the following output in .txt:
{\obeylines\obeyspaces
\texttt{
\input{./Problem3/outputs3.txt}
}}

It produces the following figures:

\begin{figure}[H]
  \centering
  \includegraphics[width=0.75\linewidth]{./Problem3/32.png}
  \caption{\textit{The analytical and numerical solution from $t=1$ to $t=1000$ years of the ODE. Initial conditions are: $D(1)=3$ and $D'(1)=2$.}}
\end{figure}

\begin{figure}[H]
  \centering
  \includegraphics[width=0.75\linewidth]{./Problem3/1010.png}
  \caption{\textit{The same as the last figure but with $D(1)=10$ and $D'(1)=-10$.}}
\end{figure}

\begin{figure}[H]
  \centering
  \includegraphics[width=0.75\linewidth]{./Problem3/50.png}
  \caption{\textit{The same as the last figure but with $D(1)=5$ and $D'(1)=0$.}}
\end{figure}

\section{Zeldovich approximation}

Here we look at the answers to Problem 4 from the second Hand-in set. The general functions that I use are coded up in a more general file given before the solutions.\\


Our main functions file is given by:
\lstinputlisting{./Problem4/functions24.py}

We now continue to the answers to the subquestions:

\subsection*{4ab}

I implemented the romberg integration algorithm and the Ridders method for differentiation.\\
Wolfram Alpha gives as result for $D(z=50)=0.019607780428266$, and for $dD/dt=273.8054053377905$, given that result for the integral. We'll take these as the analytical values.\\
I present both a and b together because they depend quite heavily on each other.\\
The analytical time-derivative of $D(t)$ is given as:
$$\dot{D}(t)=\dfrac{5H_0}{2}\Omega_m\left(\dfrac{-3\Omega_m a^{-3}}{2}\int\limits_z^{\infty}\dfrac{1+z'}{H^3(z')}dz'+\dfrac{1}{a^2*H(z)}\right).$$

The code used for running the problem is:
\lstinputlisting{./Problem4/Handin24ab.py}

It produced the following output in .txt:
{\obeylines\obeyspaces
\texttt{
\input{./Problem4/outputs4ab.txt}
}}

Thus we have managed to get within the $10^{-5}$ accuracy asked. We did however not manage to get the asked accuracy for the derivative. This could be due to the fact that we perform an integration within the function that has to find the derivative. Furthermore it could be due to Ridders algorithm not being suitable here, because of weird artefacts in the function. I did not take the time to research this into too much detail.

\subsection*{4c}

We start by creating a particle mesh and getting the $S(q)$ from Gaussian Random Fields.\\
Then we use the described update-rule to get and plot new positions for each $a$ in $\{0.0025,....,1\}$.

The code used for running the problem is:
\lstinputlisting{./Problem4/Handin24c.py}

It produced the following output in .txt:
{\obeylines\obeyspaces
\texttt{
\input{./Problem4/outputs4c.txt}
}}

The code produced a movie contained in the folder for Problem 4.\\

It produces the following figures:

\begin{figure}[H]
  \centering
  \includegraphics[width=0.75\linewidth]{./Problem4/2dposx.png}
  \caption{\textit{The x-position of the first ten particles along the y-axis (all starting at $x=0$).}}
\end{figure}

\begin{figure}[H]
  \centering
  \includegraphics[width=0.75\linewidth]{./Problem4/2dposy.png}
  \caption{\textit{The y-position of the first ten particles along the y-axis (all starting at $x=0$ and $y=0,1,2,3,4,5,6,7,8,9$). The jumps occur due to the periodic boundary conditions.}}
\end{figure}

\begin{figure}[H]
  \centering
  \includegraphics[width=0.75\linewidth]{./Problem4/2dmom.png}
  \caption{\textit{The absolute value of the momentum for the first 10 particles.}}
\end{figure}

\begin{figure}[H]
  \centering
  \includegraphics[width=0.75\linewidth]{./Problem4/2dmomx.png}
  \caption{\textit{The value of the x-momentum for the first 10 particles.}}
\end{figure}

\begin{figure}[H]
  \centering
  \includegraphics[width=0.75\linewidth]{./Problem4/2dmomy.png}
  \caption{\textit{The value of the y-momentum for the first 10 particles.}}
\end{figure}

I have no clue how to interpret these figures, but I am astonished by the enormous values that the momenta assume in both directions.\\
I think this could be due to less accurate integration (I brought down the accuracy here w.r.t. 4a and b, which I did to be able to conserve time because there are many integrals to be taken in 4c and d). It also could be due to the fact that this approximation only holds for lower $a$, and not for $a\approx1$.

\subsection*{4d}

We start by creating a particle mesh and getting the $S(q)$ from Gaussian Random Fields.\\
Then we use the described update-rule to get and plot new positions for each $a$ in $\{1/51,....,1\}$.

The code used for running the problem is:
\lstinputlisting{./Problem4/Handin24d.py}

It produced the following output in .txt:
{\obeylines\obeyspaces
\texttt{
\input{./Problem4/outputs4d.txt}
}}

The code produced a movie contained in the folder for Problem 4.\\

It produces the following figures:

\begin{figure}[H]
  \centering
  \includegraphics[width=0.75\linewidth]{./Problem4/3dposx.png}
  \caption{\textit{The x-position of the first ten particles along the z-axis (all starting at $x=0$).}}
\end{figure}

\begin{figure}[H]
  \centering
  \includegraphics[width=0.75\linewidth]{./Problem4/3dposy.png}
  \caption{\textit{The y-position of the first ten particles along the z-axis (all starting at $x=0$ and $y=0$). The jumps occur due to the periodic boundary conditions.}}
\end{figure}

\begin{figure}[H]
  \centering
  \includegraphics[width=0.75\linewidth]{./Problem4/3dposz.png}
  \caption{\textit{The z-position of the first ten particles along the z-axis (all starting at $x=0$ and $y=0$ and $z=0,1,2,3,4,5,6,7,8,9$). The jumps occur due to the periodic boundary conditions.}}
\end{figure}

\begin{figure}[H]
  \centering
  \includegraphics[width=0.75\linewidth]{./Problem4/3dmom.png}
  \caption{\textit{The absolute value of the momentum for the first 10 particles.}}
\end{figure}

\begin{figure}[H]
  \centering
  \includegraphics[width=0.75\linewidth]{./Problem4/3dmomx.png}
  \caption{\textit{The value of the x-momentum for the first 10 particles.}}
\end{figure}

\begin{figure}[H]
  \centering
  \includegraphics[width=0.75\linewidth]{./Problem4/3dmomy.png}
  \caption{\textit{The value of the y-momentum for the first 10 particles.}}
\end{figure}

\begin{figure}[H]
  \centering
  \includegraphics[width=0.75\linewidth]{./Problem4/3dmomz.png}
  \caption{\textit{The value of the z-momentum for the first 10 particles.}}
\end{figure}


\section{Mass assignment schemes}

Here we look at the answers to Problem 5 from the second Hand-in set. The general functions that I use are coded up in a more general file given before the solutions.\\

I did implement a bit-reversal, but the way I implemented my FFT it seemed to not be needed in the end.

Our main functions file is given by:
\lstinputlisting{./Problem5/functions25.py}

We now continue to the answers to the subquestions:

\subsection*{5ab}

If the shape is a delta function with a given mass, and we take the mass and size to be of unit value (physical unit not important) we get the results found below. I combined 5a and b because they depend on each other.\\

To find the grid point closest to a particle we add $.5$ and make an int out of it, because that rounds it down always. We thus assign the mass of the particle to the grid point closest to it by adding 1 to the cell signified by the hereby found coordinates.\\

The code used for running the problem is:
\lstinputlisting{./Problem5/Handin25ab.py}

It produced the following output in .txt:
{\obeylines\obeyspaces
\texttt{
\input{./Problem5/outputs5ab.txt}
}}

It produced the following figures:
\begin{figure}[H]
  \centering
  \includegraphics[width=0.75\linewidth]{./Problem5/a4.png}
  \caption{\textit{The slice for $z=4$ of the NGP-method mesh.}}
\end{figure}

\begin{figure}[H]
  \centering
  \includegraphics[width=0.75\linewidth]{./Problem5/a9.png}
  \caption{\textit{The slice for $z=4$ of the NGP-method mesh.}}
\end{figure}

\begin{figure}[H]
  \centering
  \includegraphics[width=0.75\linewidth]{./Problem5/a11.png}
  \caption{\textit{The slice for $z=4$ of the NGP-method mesh.}}
\end{figure}

\begin{figure}[H]
  \centering
  \includegraphics[width=0.75\linewidth]{./Problem5/a14.png}
  \caption{\textit{The slice for $z=4$ of the NGP-method mesh.}}
\end{figure}

\begin{figure}[H]
  \centering
  \includegraphics[width=0.75\linewidth]{./Problem5/b4.png}
  \caption{\textit{The value of the bin for the 4th cell for one particle at all possible positions.}}
\end{figure}

\begin{figure}[H]
  \centering
  \includegraphics[width=0.75\linewidth]{./Problem5/b0.png}
  \caption{\textit{The value of the bin for the 0th cell for one particle at all possible positions.}}
\end{figure}

The last two figures look as I'd have expected given the method we use (and the periodic boundary conditions).

\subsection*{5c}

We assign mass now by checking which parts of the cloud are closest to which grid points. This can be done by getting the floor values of the coordinates of a given particle (this will be the bottom left corner of the cell containing the particle), and finding the distance along each direction from the particle to that floor-point.\\
These are the height, depth and length of the 'subbox' of the box around the particle that is closest to the grid point opposite to the bottom left corner.\\
(This is way more easily explained graphically....).\\
In the same way we can assign mass to the other grid points along this cell.

The code used for running the problem is:
\lstinputlisting{./Problem5/Handin25c.py}

It produced the following output in .txt:
{\obeylines\obeyspaces
\texttt{
\input{./Problem5/outputs5c.txt}
}}

It produced the following figures:
\begin{figure}[H]
  \centering
  \includegraphics[width=0.75\linewidth]{./Problem5/c4.png}
  \caption{\textit{The slice for $z=4$ of the CIC-method mesh.}}
\end{figure}

\begin{figure}[H]
  \centering
  \includegraphics[width=0.75\linewidth]{./Problem5/c9.png}
  \caption{\textit{The slice for $z=4$ of the CIC-method mesh.}}
\end{figure}

\begin{figure}[H]
  \centering
  \includegraphics[width=0.75\linewidth]{./Problem5/c11.png}
  \caption{\textit{The slice for $z=4$ of the CIC-method mesh.}}
\end{figure}

\begin{figure}[H]
  \centering
  \includegraphics[width=0.75\linewidth]{./Problem5/c14.png}
  \caption{\textit{The slice for $z=4$ of the CIC-method mesh.}}
\end{figure}

\begin{figure}[H]
  \centering
  \includegraphics[width=0.75\linewidth]{./Problem5/c4c.png}
  \caption{\textit{The value of the bin for the 4th cell for one particle at all possible positions.}}
\end{figure}

\begin{figure}[H]
  \centering
  \includegraphics[width=0.75\linewidth]{./Problem5/c0c.png}
  \caption{\textit{The value of the bin for the 0th cell for one particle at all possible positions.}}
\end{figure}

These first figures look like a smoother counterpart to the figures from 5a, which was to be expected.\\
The last two figures again look as I'd have expected given the method we use (and the periodic boundary conditions).


\subsection*{5d}

I sort of implemented the algorithm from the slides, but it did not seem to work, so I changed a few things, and then did not need a bit-reversal anymore. The results are in agreement with analytical soltions and the scipy fftpack, so I do not know what is going right/wrong.\\

The code used for running the problem is:
\lstinputlisting{./Problem5/Handin25d.py}

It produced the following output in .txt:
{\obeylines\obeyspaces
\texttt{
\input{./Problem5/outputs5d.txt}
}}

It produced the following figures:
\begin{figure}[H]
  \centering
  \includegraphics[width=0.75\linewidth]{./Problem5/1d.png}
  \caption{\textit{FFT of $f(x)=\sin(2\pi x/5)$ for my implementation, scipy.fftpack and the analytical value. I used 64 values between $0$ and 20 to take the FFT.}}
\end{figure}

Because of the log-y-scale you do not see the $0$ amplitudes of the theoretical FFT. You can see that the FFTs are all in agreement. (Except the numerical FFT's are not as defined as the theoretical one).

\subsection*{5e}

By generalizing the 1d FFT from 5d we get a 2- and 3D-FFT for use in this exercise.\\
I generalized the sine from 5d for 2D and implemented a multvariate Gaussian (of which we expect just a multivariate Gaussian as FFT again).

The code used for running the problem is:
\lstinputlisting{./Problem5/Handin25e.py}

It produced the following output in .txt:
{\obeylines\obeyspaces
\texttt{
\input{./Problem5/outputs5e.txt}
}}

It produced the following figures:
\begin{figure}[H]
  \centering
  \includegraphics[width=0.75\linewidth]{./Problem5/e2down.png}
  \caption{\textit{FFT of $f(x,y)=\sin(2\pi (x+y)/5)$ for my implementation.}}
\end{figure}

\begin{figure}[H]
  \centering
  \includegraphics[width=0.75\linewidth]{./Problem5/e2dscipy.png}
  \caption{\textit{FFT of $f(x,y)=\sin(2\pi (x+y)/5)$ for scipy.fftpack.}}
\end{figure}

The theoretical FFT is equal to only the two yellow points shown in the figures above.

\begin{figure}[H]
  \centering
  \includegraphics[width=0.75\linewidth]{./Problem5/e3dxzscipy.png}
  \caption{\textit{FFT of a multivariate Gaussian in 3D in the $x-z$-plane for scipy.fftpack.}}
\end{figure}

\begin{figure}[H]
  \centering
  \includegraphics[width=0.75\linewidth]{./Problem5/e3dxyscipy.png}
  \caption{\textit{FFT of a multivariate Gaussian in 3D in the $x-y$-plane for scipy.fftpack.}}
\end{figure}

\begin{figure}[H]
  \centering
  \includegraphics[width=0.75\linewidth]{./Problem5/e3dyzscipy.png}
  \caption{\textit{FFT of a multivariate Gaussian in 3D in the $y-z$-plane for scipy.fftpack.}}
\end{figure}

\begin{figure}[H]
  \centering
  \includegraphics[width=0.75\linewidth]{./Problem5/e3dxzown.png}
  \caption{\textit{FFT of a multivariate Gaussian in 3D in the $x-z$-plane for my own implementation.}}
\end{figure}

\begin{figure}[H]
  \centering
  \includegraphics[width=0.75\linewidth]{./Problem5/e3dxyown.png}
  \caption{\textit{FFT of a multivariate Gaussian in 3D in the $x-y$-plane for my own implementation.}}
\end{figure}

\begin{figure}[H]
  \centering
  \includegraphics[width=0.75\linewidth]{./Problem5/e3dyzown.png}
  \caption{\textit{FFT of a multivariate Gaussian in 3D in the $y-z$-plane for my own implementation.}}
\end{figure}

As said above this is as we expected.



\subsection*{5fg}

We apply the FFT-algorithm to the mesh created in c. We create a $k_x,k_y,k_z$-array to find $k^2$ for all the points and divide. Then we IFFT (only difference is $i-->-i$ in the exponential for this) to find the potential up to a constant (which should include the normalization factor needed for the IFFT).\\
To deal with $k^2=0$ for $k_x=k_y=k_z=0$ I set that value to $1$ instead.\\
The potential has enormous values, but is defined up to a constant, although I do not trust that I need to do it this way.

The code used for running the problem is:
\lstinputlisting{./Problem5/Handin25fg.py}

It produced the following output in .txt:
{\obeylines\obeyspaces
\texttt{
\input{./Problem5/outputs5fg.txt}
}}
These are the ten gradients of the first ten randomly generated positions from a.

It produced the following figures:
\begin{figure}[H]
  \centering
  \includegraphics[width=0.75\linewidth]{./Problem5/f4.png}
  \caption{\textit{Slice of the potential according to (16) of the mesh from 5c at $z=4$.}}
\end{figure}

\begin{figure}[H]
  \centering
  \includegraphics[width=0.75\linewidth]{./Problem5/f9.png}
  \caption{\textit{Slice of the potential according to (16) of the mesh from 5c at $z=9$.}}
\end{figure}

\begin{figure}[H]
  \centering
  \includegraphics[width=0.75\linewidth]{./Problem5/f11.png}
  \caption{\textit{Slice of the potential according to (16) of the mesh from 5c at $z=11$.}}
\end{figure}

\begin{figure}[H]
  \centering
  \includegraphics[width=0.75\linewidth]{./Problem5/f14.png}
  \caption{\textit{Slice of the potential according to (16) of the mesh from 5c at $z=14$.}}
\end{figure}

\begin{figure}[H]
  \centering
  \includegraphics[width=0.75\linewidth]{./Problem5/f7yz.png}
  \caption{\textit{Slice of the potential according to (16) of the mesh from 5c at $x=7$.}}
\end{figure}

\begin{figure}[H]
  \centering
  \includegraphics[width=0.75\linewidth]{./Problem5/f7xz.png}
  \caption{\textit{Slice of the potential according to (16) of the mesh from 5c at $y=7$.}}
\end{figure}


\section{Classifying $\gamma$-ray bursts}

Here we look at the answers to Problem 6 from the second Hand-in set. The general functions that I use are coded up in a more general file given before the solutions.\\

I just implemented all needed ingredients for logistic regression as can be found in Lecture 9.\\

We deleted the two rows where $T_{90}$ was missing and set all data with $T_{90}\geq10$ as being a long $\gamma$-ray burst.\\

To deal with missing data we set $-1$'s to $0$ if the data is not logarithmic, and if they are we take the exponent and then set value that previously were $-1$ equal to $0$.\\
This should in theory not introduce biases and make it able to get as much data into the model as possible.\\

After some testing we found that the best combination of datacolumns to use was that with redshift, the log of the mass, log of the metallicity and the specific star formation rate.\\

We take a learning parameter as $\alpha=0.309$ and find the output stated below.

Our main functions file is given by:
\lstinputlisting{./Problem6/functions26.py}

It produced the following output in .txt:
{\obeylines\obeyspaces
\texttt{
\input{./Problem6/outputs6.txt}
}}
This shows the optimal parameters for the regression given the datacolumns that we take into account.\\
It shows the final accuracy and the amount of iterations before reaching that. These thus are parameters such that we have:
$$\theta_0+\theta_1z+\theta_2(M/M_{\odot})+\theta_3(Z/Z_{\odot})+\theta_4SSFR.$$

It produced the following figures:
\begin{figure}[H]
  \centering
  \includegraphics[width=0.75\linewidth]{./Problem6/6.png}
  \caption{\textit{The predicted value from logistic regression (0 (and blue) for short and 1 (and orange) for long) for the datapoints. The blue line signifies the threshold that separates the classes.}}
\end{figure}

As can be seen, it produces two false negatives, 6 true negatives, and many false positives. But, it does give a somewhat better result than just stating True for each datapoint, so it had some effect...\\
Due to the positions of the datapoints and the many missing datapoints, I did not expect higher orders to make a big difference.


\section{Building a quadtree}

Here we look at the answers to Problem 7 from the second Hand-in set. The general functions that I use are coded up in a more general file given before the solutions.\\
I coded up a class for the node and for the tree respectively, so actions for the entire tree can be performed with one command, and the nodes can store all needed info.\\
We take the mass to be the one assigned to each particle in the colliding.hdf5-file.


Our main functions file is given by:
\lstinputlisting{./Problem7/functions27.py}

It produced the following output in .txt:
{\obeylines\obeyspaces
\texttt{
\input{./Problem7/outputs7.txt}
}}
These are the $n=0$ multipole moments of each node (bottom to top) containing $i=100$.


It produced the following figures:
\begin{figure}[H]
  \centering
  \includegraphics[width=0.75\linewidth]{./Problem7/QT.png}
  \caption{\textit{The quadtree for colliding.hdf5.}}
\end{figure}



\section{Failed}
I did not get within the asked accuracy for the derivative in 4(b).\\
While trying to collect all results and setting up the bash file the results for question $4d$ seemed to be wrong suddenly. Therefore I altered the normalization factor to have the movie show a bit more action (the factor needs to be $64^{3/2}$ instead of $64^2$. I also plotted the 3D-slices in one figure to spare some time.\\
I do not expect exercise 5f and 5g to have been done correctly, but thought it would be a shame to not try to implement them quickly.\\
I consider exercise 6 to be failed a bit, because the results are not astonishing. The logistic regression code used seemed to work fine with the Tutorial Exercise.\\
I did not make exercise 8.\\

\end{document}

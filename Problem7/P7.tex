\section{Building a quadtree}

Here we look at the answers to Problem 7 from the second Hand-in set. The general functions that I use are coded up in a more general file given before the solutions.\\
I coded up a class for the node and for the tree respectively, so actions for the entire tree can be performed with one command, and the nodes can store all needed info.\\
We take the mass to be the one assigned to each particle in the colliding.hdf5-file.


Our main functions file is given by:
\lstinputlisting{./Problem7/functions27.py}

It produced the following output in .txt:
{\obeylines\obeyspaces
\texttt{
\input{./Problem7/outputs7.txt}
}}
These are the $n=0$ multipole moments of each node (bottom to top) containing $i=100$.


It produced the following figures:
\begin{figure}[H]
  \centering
  \includegraphics[width=0.75\linewidth]{./Problem7/QT.png}
  \caption{\textit{The quadtree for colliding.hdf5.}}
\end{figure}

